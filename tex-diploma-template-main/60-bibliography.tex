\section*{СПИСОК ИСПОЛЬЗОВАННЫХ ИСТОЧНИКОВ}
\addcontentsline{toc}{section}{СПИСОК ИСПОЛЬЗОВАННЫХ ИСТОЧНИКОВ}

\begingroup
\renewcommand{\section}[2]{}
\begin{thebibliography}{}

	\bibitem{ege}
	Федеральный институт педагогических измерений – (дата обращения 2.12.2021). // [Электронный ресурс]. – Режим доступа: https://fipi.ru/ege

	\bibitem{prikaz}
	Приказ Министерства науки и высшего образования РФ от 13 августа 2021 г. N 753  – (дата обращения 2.12.2021). // [Электронный ресурс]. – Режим доступа: https://www.garant.ru/products/ipo/prime/doc/402694290/

	\bibitem{porydok}
	Порядок приема на обучение по программам высшего образования  – (дата обращения 2.12.2021). // [Электронный ресурс]. – Режим доступа: https://docs.cntd.ru/document/420216602

	\bibitem{celevoi}
	Постановление Правительства РФ от 13 октября 2020 г. № 1681 “О целевом обучении по образовательным программам среднего профессионального и высшего образования”  – (дата обращения 3.12.2021). // [Электронный ресурс]. – Режим доступа: https://www.garant.ru/products/ipo/prime/doc/74665624/

	\bibitem{prikaz1}
	Приказ Министерства науки и высшего образования РФ об утверждении Порядка приёма на обучение по образовательным программам высшего образования от 14 сентября 2020 г. – (дата обращения 3.12.2021). // [Электронный ресурс]. – Режим доступа: https://cdnimg.rg.ru/pril/article/195/62/75/2_5411494948547594417.pdf

	\bibitem{stat}
	Российский Статистический Ежегодник 2019 – (дата обращения 10.12.2021). // [Электронный ресурс]. – Режим доступа: https://gks.ru/bgd/regl/b19\_13/Main.htm

	\bibitem{hse}
	Мониторинг качества приёма в вузы – (дата обращения 1.11.2021). // [Электронный ресурс]. – Режим доступа: https://ege.hse.ru

	\bibitem{egestat}
	Сведения о сдаче единого государственного экзамена – (дата обращения 1.11.2021). // [Электронный ресурс]. – Режим доступа: https://data.gov.ru/opendata/7710539135-ege

	\bibitem{miccedu}
	Информационно-аналитические материалы по результатам анализа показателей эффективности образовательных организаций высшего образования – (дата обращения 1.11.2021). // [Электронный ресурс]. – Режим доступа: https://monitoring.miccedu.ru 

	\bibitem{Kotlin}
	Язык программирования Kotlin – (дата обращения 5.12.2021). // [Электронный ресурс]. – Режим доступа: https://kotlinlang.org

	\bibitem{postgresql}
	Cистема управления базами данных PostgreSQL – (дата обращения 5.12.2021). // [Электронный ресурс]. – Режим доступа: https://www.postgresql.org

	\bibitem{nasadkin}
	Насадкин М.Ю., Агентное моделирование поведения абитуриентов при выборе вуза в России / Насадкин М.Ю., Питухин Е.А., Астафьева М.П. // Фундаментальные исследования. – 2015. – № 8-2. – С. 307-311.

	\bibitem{abankina}
	Абанкина И.В. Модель многоступенчатого выбора для прогнозирования спроса на высшее образование /
	И.В. Абанкина и др. // Университетское управление:
	практика и анализ. – 2014. – № 4–5. – С. 84-94.

	\bibitem{kiselgof}
	Кисельгоф С.А. Выбор вузов абитуриентами с квадратичной функцией полезности: препринт WP7/2011/01;
	Высшая школа экономики. – М.: Изд. дом Высшей школы экономики, 2011. – 44 с.

	\bibitem{pityxin}
	Питухин Е.А. Анализ межрегиональной мобильности выпускников школ при поступлении в высшие учебные заведения 
	/ Е.А. Питухин, А.А. Семенов // Университетское управление: практика и анализ. – 2011. – № 3. – С. 82-89.

	\bibitem{praxov}
	Прахов И.А. Модель выбора вуза в условиях ЕГЭ и роль ожиданий абитуриентов: препринт WP10/2010/06;
	Гос. ун-т – Высшая школа экономики. – М.: Изд. дом Гос. ун-та – Высшей школы экономики, 2010. – 56 с.

	\bibitem{evdonin}
	Евдонин Г. А. Математическое моделирование и управление социально-экономическими 
	и политическими процессами : учеб. пособие / Г. А. Евдонин. СПб. : Издательство СЗИУ 
	РАНХиГС, 2012. 322 с.

	\bibitem{richenkov}
	Рыченков М. В. Исследование факторов, оказывающих влияние на выбор вуза абитуриентами, на различных этапах процесса поступления / Рыченков М. В., Рыченкова И. В., 
	Киреев В. С. // Современные проблемы науки и образования. М., 2013. № 6.

	\bibitem{timoxovich}
	Тимохович А. Н. Российский абитуриент вуза в условиях неопределенности // Вестник 
	Университета (Государственный университет управления). М., 2012. № 1. С. 181–185.

	\bibitem{teplov}
	Теплов Е. В., Филинова И. М. Факторы выбора абитуриентом образовательного учреждения // Среднее профессиональное образование. М., 2013. № 10. С. 43–44.

	\bibitem{bereza}
	Берёза А.Н. Поддержка принятия решения при планировании набора абитуриентов в вузе на основе нечетких моделей/ Берёза А.Н., Ершова Е.А. // Известия ЮФУ. - Технические науки. - 2011. - №7. – С. 131-136.

	\bibitem{novosadova}
	Новосадова Н.О. Моделирование приемной кампании вузов с различным качеством и реализация модели в программной среде matlab // Фундаментальные и прикладные исследования в современной науке: сборник статей II Международной научно-практической конференции (14 апреля 2018 г., г. Самара). - Самара: ЦНИК, 2018. – С. 3-6.

	\bibitem{gale}
	Gale D., Shapley L.S. College Admission and the Stability of Marriage. The American Mathematical Monthly. –
	Vol. 69. – № 1 (Jan., 1962). – P. 9-15.

\end{thebibliography}
\endgroup

\pagebreak