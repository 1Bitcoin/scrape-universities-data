\section*{ВВЕДЕНИЕ}
\addcontentsline{toc}{section}{ВВЕДЕНИЕ}

В настоящее время изучение выбора абитуриентом высшего учебного заведения остается актуальным. Выбирая место для получения образования, абитуриент должен решить какие специальности для него предпочтительны, какие характеристики вуза играют ключевую роль.

Неразрывно с этим выбором связана и политика вуза, направленная на привлечение лучших абитуриентов. Для достижения этой цели применяются различные методы, одним из которых является моделирование приёмной кампании, позволяющее выявить слабые стороны процесса приёма, а также
изучить реакцию абитуриентов на закрытие, либо открытие новой специальности, изменение экономического показателя региона и спрогнозировать спрос на высшее образование. 

Цель данной работы – предложить способ реализации метода прогнозирования итогов приёма в ВУЗы России на основе агентного моделирования.

Для достижения поставленной цели необходимо решить следующие задачи:

\begin{itemize}[leftmargin=1.6\parindent]
	\item[---] изучить правила организации приёма в ВУЗы России;
	\item[---] получить информацию и статистику о ВУЗах России;
	\item[---] получить информацию по УГСН и связанных с ними предметам ЕГЭ;
	\item[---] выделить наиболее важные факторы для разработки модели поведения абитуриентов;
	\item[---] изучить существующие методы и механизмы зачисления абитуриентов;
	\item[---] разработать имитационную модель поведения абитуриентов;
	\item[---] разработать метод прогнозирования итогов приёма в ВУЗы, используя модель поведения и статистические данные;
	\item[---] программно реализовать разработанный метод;
	\item[---] исследовать работу метода на различных входных данных.
\end{itemize}


\pagebreak