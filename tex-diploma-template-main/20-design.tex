\section{Конструкторская часть}

\subsection{Разработка алгоритма поиска общего названия ВУЗа}

Поскольку количество анализируемых ВУЗов составляет несколько сотен, сбор данных должен производиться в автоматизированном режиме.

В процессе анализа статистических данных было обнаружено, что именования ВУЗов на аналитических ресурсах различны, что затрудняет их объединение, поскольку название ВУЗа в данном случае является его единственным уникальным идентификатором. 

Одним из вариантов решения данной задачи является отправка двух запросов в сеть Интернет, где в качестве параметра поиска передавать названия ВУЗа с информационных источников. Таким образом, будут получены соответствия для двух различных названий ВУЗов и одного унифицированного, с помощью которого можно производить объединение данных.

В таблице \ref{list:namevuz} представлен пример такой интерпретации.

\begin{table}[!hbp]
	\caption{\label{list:namevuz}Интерпретация названия ВУЗа поисковой системой Яндекс.}
	\begin{center}
		\begin{tabular}{|c|c|}
			\hline
			\begin{tabular}[c]{@{}c@{}}Название \\ ВУЗа \\ на первом\\ ресурсе\end{tabular} & Моск. гос. техн. ун-т. им. Н.Э. Баумана                                                                                                                                                                                                                      \\ \hline
			\begin{tabular}[c]{@{}c@{}}Название \\ ВУЗа\\ на втором\\ ресурсе\end{tabular}  & \begin{tabular}[c]{@{}c@{}}федеральное государственное бюджетное \\ образовательное учреждение высшего образования \\ «Московский государственный технический \\ университет имени Н.Э.Баумана (национальный \\ исследовательский университет)»\end{tabular} \\ \hline
			\begin{tabular}[c]{@{}c@{}}Общее \\ название\end{tabular}                       & \begin{tabular}[c]{@{}c@{}}Московский государственный технический \\ университет им. Н.Э. Баумана\end{tabular}                                                                                                                                               \\ \hline
		\end{tabular}
	\end{center}
\end{table}

\subsection{Разработка алгоритма приведения данных к официальному перечню направлений подготовки высшего образования}

В данных об укрупнённых группах специальностей и направлений, полученных с мониторинга качества приёма в ВУЗы\cite{miccedu}, существует допущение — указаны неофициальные укрупнённые группы, составленные самим мониторингом из официальных групп. Для решения данной задачи необходимо выделить таблицу, в которой будет указано, какую долю имеет официальная группа, в неофициальной. С помощью данной таблицы можно пересчитать количество бюджетных мест по формуле 

\begin{equation}
N_{budget} = \sum P_{i} k_{i},
\end{equation}
\begin{tabular}{llll}
    где & $N_{budget}$ & {---} & количество бюджетных мест в официальной группе; \\
    & $P_{i}$ & {---} & \begin{tabular}[t]{@{}l@{}}количество мест в i-ой неофициальной группе;\end{tabular} \\
    & $k_{i}$ & {---} & \begin{tabular}[t]{@{}l@{}} i-я доля официальной группы в неофициальной.\end{tabular}
\end{tabular} \\

Средний балл в каждой группе ВУЗа из информационно-аналитических материалов по следующей формуле

\begin{equation}
B_{aver.score} = \frac{\sum P_{i} k_{i} B_{i}}{N_{budget}},
\end{equation}
\begin{tabular}{llll}
    где & $B_{aver.score}$ & {---} & средний балл в официальной группе; \\
    & $B_{i}$ & {---} & \begin{tabular}[t]{@{}l@{}}средний балл в i-ой неофициальной группе.\end{tabular} \\
\end{tabular} \\

Последним этапом приведения будет масштабирование значений количества бюджетных мест и среднего балла в группе. Его необходимо проводить, если верно следующее:

\begin{equation}
N_{offical} \neq  N_{unoffical},
\end{equation}
\begin{tabular}{llll}
    где & $N_{offical}$ & {---} & сумма бюджетных мест всех  официальных групп ВУЗа; \\
    & $N_{unoffical}$ & {---} & \begin{tabular}[t]{@{}l@{}}сумма бюджетных мест всех  неофициальных групп ВУЗа.\end{tabular} \\
\end{tabular} \\

В данном случае необходимо пересчитать количество бюджетных в каждой группе по описанной выше формуле:

\begin{equation}
N_{budget} = \mu \sum P_{i} k_{i},
\end{equation}
\begin{tabular}{llll}
    где & $\mu = \frac{N_{offical}}{N_{unoffical}}$ \\
\end{tabular} \\


Формула для пересчёта среднего балла остается без изменений.


\subsection{Генерация популяции агентов}

Генерация популяции агентов осуществляется с помощью следующих параметров:

\begin{itemize}[leftmargin=1.6\parindent]
	\item[---] процентные соотношения численности категорий студентов с различными диапазонами полученных результатов ЕГЭ;
	\item[---] диапазон результатов ЕГЭ, которые будут заданы агентам каждой категории;
	\item[---] данные о распределении агентов по регионам;
	\item[---] данные об связи УГСН и принимаемых предметах ЕГЭ;
	\item[---] процентные соотношения признака смены региона для каждой категории;
	\item[---] количество различных УГСН, которые рассматривает агент;
	\item[---] минимальные баллы ЕГЭ по каждому предмету.
\end{itemize}

Все агенты разделены на 4 категории: не набравшие минимальный балл, сдавшие ЕГЭ в диапазоне от минимального балла до заданного и 2 категории для конфигурирования агентов, которые сдали ЕГЭ выше результатов двух предыдущих категорий.

В каждой категории, кроме той, где абитуриенты на набрали минимальный балл, агенту проставляются результаты успешно сданных ЕГЭ, а также интересующие его УГСН. Для этого при обработке очередного агента случайным образом определяется его специализация, в зависимости от чего и прописываются результаты ЕГЭ и связанные с ними интересы. Всего таких специализаций 8 и соответствующие им ЕГЭ следующие:

\begin{itemize}[leftmargin=1.6\parindent]
	\item[---] иностранный язык и литература;
	\item[---] история и обществознание;
	\item[---] химия и биология;
	\item[---] информатика и ИКТ;
	\item[---] физика;
	\item[---] физика и информатика и ИКТ;
	\item[---] биология, химия и информатика и ИКТ;
	\item[---] история, обществознание, география и литература.
\end{itemize}

Следует уточнить, что русский язык и математика являются обязательными к сдаче, поэтому они не указывались в списке выше, поскольку уже подразумеваются как базовые.

Первым этапом генерации популяции агентов является создание пустого агента без характеристик и проставление ему домашнего региона и признака готовности к смене домашнего региона в зависимости от параметров конфигурации, полученной от пользователя. Домашний регион определяется из полученного распределения абитуриентов по регионам. Текущий обрабатываемый регион будет для созданного агента домашним.

Вторым этапом является установка каждому агенту результатов сдачи ЕГЭ, в зависимости от специализации, и связанных с ними интересов. Результаты сдачи ЕГЭ проставляются каждому агенту случайным образом из заданного диапазона значений для конкретной категории. Из всех подходящих агенту УГСН выбирается то количество, которое указано в конфигурации, полученной от пользователя.

Полученная популяция агентов сохраняется в базу данных для дальнейшего использования для моделирования.

\subsection{Модель поведения и характеристики агента}

В данной задаче агентами являются абитуриенты – сущности, обладающие активностью и характеристиками.

Базовыми характеристиками абитуриента являются:

\begin{itemize}[leftmargin=1.6\parindent]
	\item[---] уникальный идетификатора абитуриента;
	\item[---] возможность смены региона;
	\item[---] результаты сдачи ЕГЭ;
	\item[---] интересующие УГСН, определяющиеся по сданным ЕГЭ.

\end{itemize}

Каждый абитуриент оценивает свою ситуацию, анализируя свое положение в конкурсном списке (отсортированном по сумме баллов ЕГЭ) и по количеству доступных бюджетных мест на данный УГСН. Если он не попадает в доступные места, то принимается решение о полной смене текущего ВУЗа со всеми заявлениями в нем или о поиске другого ВУЗа, в который можно положить заявление на такой же УГСН.

Оценивание ситуаций производится до тех пор, пока можно подавать новые заявления. Затем абитуриент должен положить оригинал аттестата к одному из своих заявлений. Искомое заявление выбирается исходя из престижности ВУЗа – абитуриенты стремятся попасть в наиболее престижные места. Если выбор производится внутри одного ВУЗа на разных УГСН, то сравниваются величины престижности данных УГСН. 

На этапе выбора заявления для подачи оригинала аттестата возможна ситуация, когда абитуриент не имеет ни одного заявления с копией аттестата, соответственно он не может претендовать на возможность обучения в ВУЗе.

\subsection{Разработка метод прогнозирования итогов приема в ВУЗы}

Ниже представлена IDEF0-диаграмма разрабатываемого метода на рисунке  \ref{A-0}.

\begin{figure}[hbtp]
	\centering
	\includegraphics[scale=0.8]{idef0/01\_A-0.pdf}
	\caption{IDEF0 диаграмма ветка A-0}
	\label{A-0}
\end{figure}

Разрабатываемый метод прогнозирования состоит из четырех этапов:

\begin{itemize}[leftmargin=1.6\parindent]
	\item[---] базовое распределение абитуриентов по ВУЗам на основе проходного балла прошлого года;
	\item[---] перераспределение агентами поданных заявлений для поиска УГСН, на который имеются шансы пройти по бюджетным местам в текущий момент;
	\item[---] подача оригинала аттестата к своему заявлению, по результатам сравнения престижности ВУЗов и их УГСН;
	\item[---] анализ полученных результатов.
\end{itemize}

\begin{figure}[hbtp]
	\centering
	\includegraphics[scale=0.8]{idef0/02\_A0.pdf}
	\caption{IDEF0 диаграмма ветка A0}
	\label{A0}
\end{figure}

Настройки конфигурации включают в себя поля:

\begin{itemize}[leftmargin=1.6\parindent]
	\item[---] год проведения моделирования;
	\item[---] количество абитуриентов, участвующих в моделировании;
	\item[---] длительность этапа перераспределения заявлений с копиями аттестатов;
	\item[---] максимальное количество ВУЗов, в которое может подать заявления агент;
	\item[---] длительность этапа поиска наилучшего места для оригинала аттестата.

\end{itemize}

Первым этапом моделирования явлется базовое распределение абитуриентов по УГСН в ВУЗах на основе среднего балла по УГСН в прошлом году. Абитуриенты просматривают ВУЗы, начиная с самого престижного, проверяют наличие интересующего УГСН, какие результаты ЕГЭ необходимы для подачи заявления, сравнивают со своими результатами (средним баллом по предметам, необходимым для поступления на данный УГСН) и кладут копию аттестата к заявлению.

Второй этап моделирования представлен в виде схемы алгоритма поиска подходящих УГСН на рисунке \ref{scheme:find}

\begin{figure}[hbtp]
	\centering
	\includegraphics[scale=0.6]{idef0/find.pdf}
	\caption{Схема алгоритма поиска подходящего УГСН}
	\label{scheme:find}
\end{figure}


Третий этап – подача оригинала аттестата к своему заявлению. Студенты, не имеющие заявлений ни в каком ВУЗе не рассматриваются. Анализируем каждое заявление абитуриента, проходит ли он в текущий момент по бюджетным местам среди других абитуриентов с оригиналами аттестатов. Отмечаем данную заявку как лучшую, если абитуриент проходит на нее и она лучше предыдущего выбора. Сравнение текущего и предыдущего выбора производится на основе престижности ВУЗа, если сравниваются УГСН в разных ВУЗах и престижности УГСН, если сравнение производится внутри одного ВУЗа.

Завершающий этап - зачисления и анализ сложившейся ситуации в конкурсном списке каждого УГСН, вычисление минимального, максимального, среднего балла по данному УГСН и ВУЗу в целом и сравнение результата моделирования с итогами приема абитуриентов предыдущего года. Зачисление на доступные бюджетные места абитуриентов с наиболее высокими баллами в конкурсном списке.

\subsection{Особенности и допущения предлагаемого метода}

Данный метод не предусматривает многопоточную реализацию процесса моделирования. Для оптимизации работы с базой данных необходимо использовать массовую вставку записей в одной транзакции, вместо множества запросов в транзакциях. 

При выборе из нескольких УГСН, которые подходят по результатам ЕГЭ не учитываются индивидуальные предпочтения отдельно взятого агента, предпочтение отдается наиболее престижному УГСН, на который абитуриент может поступить. 

При моделировании учитываются только бюджетные места для обучения, поскольку сведений об количестве платных мест крайне мало в открытых источниках.

Агент подает оригинал аттестата только в то место, куда гарантированно поступит, рискованные и нечестные стратегии не предусмотрены.

\subsection{Используемые структуры данных}

Для хранения списка агентов необходимо использовать стандартную реализацию массива фиксированной длины, поскольку количество студентов не меняется в процесса моделирования.

Для эффективного доступа к статистической информации о ВУЗе, применяется хэш-таблица, которая позволяет хранить пары вида ключ-значение и выполнять три операции: операцию добавления новой пары, операцию поиска и операцию удаления пары по ключу.

Кеширование значений предметов ЕГЭ для каждого УГСН производить с помощью хэш-таблицы, значением которой является множество – коллекция неупорядоченных элеметов.

Конкурсные списки с информацией на какой УГСН подал заявление агент хранить в потокобезопасной реализации хэш-таблицы для избежания ошибок при модификации одной коллекции разными итераторами.



\pagebreak