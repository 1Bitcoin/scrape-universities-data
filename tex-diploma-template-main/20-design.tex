\section{Конструкторская часть}

\subsection{Разработка алгоритма поиска общего названия ВУЗа}

Поскольку количество анализируемых ВУЗов составляет несколько сотен, сбор данных должен производиться в автоматизированном режиме.

Для сбора и преобразования статистики о российских ВУЗах в машино-читаемый формат и записи в базу данных, с целью дальнейшей обработки и анализа, был использован Kotlin — статически типизированный высокоуровневый язык программирования\cite{Kotlin}, а для работы с базой данных свободная объектно-реляционная система управления базами данных PostgreSQL\cite{postgresql}.

В процессе анализа информации было обнаружено, что именования ВУЗов на упомянутых выше аналитических ресурсах различно, что затрудняет объединение статистических данных, поскольку название ВУЗа в данном случае является его уникальным идентификатором. Для унификации названий ВУЗов была использована поисковая система от компании Яндекс. С помощью протокола http был отправлен get-запрос к поисковой системе, где в качестве параметра передавалось название вуза с информационного источника. Анализируя полученный в виде html страницы ответ, можно получить название ВУЗа в интерпретации поисковой системы. В таблице \ref{list:namevuz} представлен пример такой интерпретации.

\begin{table}[!hbp]
	\caption{\label{list:namevuz}Интерпретация названия ВУЗа поисковой системой Яндекс.}
	\begin{center}
\begin{tabular}{|c|c|}
\hline
\begin{tabular}[c]{@{}c@{}}Название \\ ВУЗа \\ на первом\\ ресурсе\end{tabular} & Моск. гос. техн. ун-т. им. Н.Э. Баумана                                                                                                                                                                                                                      \\ \hline
\begin{tabular}[c]{@{}c@{}}Название \\ ВУЗа\\ на втором\\ ресурсе\end{tabular}  & \begin{tabular}[c]{@{}c@{}}федеральное государственное бюджетное \\ образовательное учреждение высшего образования \\ «Московский государственный технический \\ университет имени Н.Э.Баумана (национальный \\ исследовательский университет)»\end{tabular} \\ \hline
\begin{tabular}[c]{@{}c@{}}Общее \\ название\end{tabular}                       & \begin{tabular}[c]{@{}c@{}}Московский государственный технический \\ университет им. Н.Э. Баумана\end{tabular}                                                                                                                                               \\ \hline
\end{tabular}
\end{center}
\end{table}

Однако при обработке названий филиалов, практически во всех случаях результатом было название головного ВУЗа, что являлось ошибочным.  

\subsection{Разработка алгоритма приведения данных к официальному перечню направлений подготовки высшего образования}

В данных об укрупнённых группах специальностей и направлений, полученных с мониторинга качества приёма в ВУЗы\cite{miccedu}, существует допущение — указаны неофициальные укрупнённые группы, составленные самим мониторингом из официальных. Для решения данной задачи необходимо выделить таблицу, в которой будет указано, какую долю имеет официальная группа, в неофициальной. С помощью данной таблицы можно пересчитать количество бюджетных мест по формуле 

\begin{equation}
N_{budget} = \sum P_{i} k_{i},
\end{equation}
\begin{tabular}{llll}
    где & $N_{budget}$ & {---} & количество бюджетных мест в официальной группе; \\
    \addlinespace
    & $P_{i}$ & {---} & \begin{tabular}[t]{@{}l@{}}количество мест в i-ой неофициальной группе;\end{tabular} \\
    \addlinespace
    & $k_{i}$ & {---} & \begin{tabular}[t]{@{}l@{}} i-я доля официальной группы в неофициальной.\end{tabular}
\end{tabular} \\

Средний балл в каждой группе ВУЗа из информационно-аналитических материалов по следующей формуле

\begin{equation}
B_{aver.score} = \frac{\sum P_{i} k_{i} B_{i}}{N_{budget}},
\end{equation}
\begin{tabular}{llll}
    где & $B_{aver.score}$ & {---} & средний балл в официальной группе; \\
    \addlinespace
    & $B_{i}$ & {---} & \begin{tabular}[t]{@{}l@{}}средний балл в i-ой неофициальной группе.\end{tabular} \\
\end{tabular} \\

Последним этапом приведения будет масштабирование значений количества бюджетных мест и среднего балла в группе. Его необходимо проводить, если верно следующее:

\begin{equation}
N_{offical} \neq  N_{unoffical},
\end{equation}
\begin{tabular}{llll}
    где & $N_{offical}$ & {---} & сумма бюджетных мест всех  официальных групп ВУЗа; \\
    \addlinespace
    & $N_{unoffical}$ & {---} & \begin{tabular}[t]{@{}l@{}}сумма бюджетных мест всех  неофициальных групп ВУЗа.\end{tabular} \\
\end{tabular} \\

В данном случае необходимо пересчитать количество бюджетных в каждой группе по описанной выше формуле:

\begin{equation}
N_{budget} = \mu \sum P_{i} k_{i},
\end{equation}
\begin{tabular}{llll}
    где & $\mu = \frac{N_{offical}}{N_{unoffical}}$ \\
\end{tabular} \\


Формула для пересчёта среднего балла остается без изменений.


\pagebreak