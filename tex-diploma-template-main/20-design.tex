\section{Конструкторская часть}

\subsection{Разработка алгоритма поиска общего названия ВУЗа}

Поскольку количество анализируемых ВУЗов составляет несколько сотен, сбор данных должен производиться в автоматизированном режиме.

В процессе анализа статистических данных было обнаружено, что именования ВУЗов на аналитических ресурсах различны, что затрудняет их объединение, поскольку название ВУЗа в данном случае является его уникальным идентификатором. 

Одним из вариантов решения данной задачи является отправка двух запросов в сеть Интернет, где в  качестве параметра поиска передавать название ВУЗа с информационного источника. Таким образом, будут получены соответствия для двух различных названий ВУЗов и одного унифицированного, с помощью которого можно производить объединение данных.

\subsection{Разработка алгоритма приведения данных к официальному перечню направлений подготовки высшего образования}

В данных об укрупнённых группах специальностей и направлений, полученных с мониторинга качества приёма в ВУЗы\cite{miccedu}, существует допущение — указаны неофициальные укрупнённые группы, составленные самим мониторингом из официальных групп. Для решения данной задачи необходимо выделить таблицу, в которой будет указано, какую долю имеет официальная группа, в неофициальной. С помощью данной таблицы можно пересчитать количество бюджетных мест по формуле 

\begin{equation}
N_{budget} = \sum P_{i} k_{i},
\end{equation}
\begin{tabular}{llll}
    где & $N_{budget}$ & {---} & количество бюджетных мест в официальной группе; \\
    \addlinespace
    & $P_{i}$ & {---} & \begin{tabular}[t]{@{}l@{}}количество мест в i-ой неофициальной группе;\end{tabular} \\
    \addlinespace
    & $k_{i}$ & {---} & \begin{tabular}[t]{@{}l@{}} i-я доля официальной группы в неофициальной.\end{tabular}
\end{tabular} \\

Средний балл в каждой группе ВУЗа из информационно-аналитических материалов по следующей формуле

\begin{equation}
B_{aver.score} = \frac{\sum P_{i} k_{i} B_{i}}{N_{budget}},
\end{equation}
\begin{tabular}{llll}
    где & $B_{aver.score}$ & {---} & средний балл в официальной группе; \\
    \addlinespace
    & $B_{i}$ & {---} & \begin{tabular}[t]{@{}l@{}}средний балл в i-ой неофициальной группе.\end{tabular} \\
\end{tabular} \\

Последним этапом приведения будет масштабирование значений количества бюджетных мест и среднего балла в группе. Его необходимо проводить, если верно следующее:

\begin{equation}
N_{offical} \neq  N_{unoffical},
\end{equation}
\begin{tabular}{llll}
    где & $N_{offical}$ & {---} & сумма бюджетных мест всех  официальных групп ВУЗа; \\
    \addlinespace
    & $N_{unoffical}$ & {---} & \begin{tabular}[t]{@{}l@{}}сумма бюджетных мест всех  неофициальных групп ВУЗа.\end{tabular} \\
\end{tabular} \\

В данном случае необходимо пересчитать количество бюджетных в каждой группе по описанной выше формуле:

\begin{equation}
N_{budget} = \mu \sum P_{i} k_{i},
\end{equation}
\begin{tabular}{llll}
    где & $\mu = \frac{N_{offical}}{N_{unoffical}}$ \\
\end{tabular} \\


Формула для пересчёта среднего балла остается без изменений.

\subsection{Характеристики агента}

В данной задаче агентами являются абитуриенты – сущности, обладающие активностью и характеристиками.

Базовыми характеристиками абитуриента являются:

\begin{itemize}[leftmargin=1.6\parindent]
	\item[---] уникальный идетификатора абитуриента;
	\item[---] возможность смены региона;
	\item[---] результаты сдачи ЕГЭ;
	\item[---] интересующие УГСН, определяющиеся по сданным ЕГЭ.

\end{itemize}


\subsection{Генерация популяции агентов}



\subsection{Модель поведения агента}

Каждый абитуриент оценивает свою ситуацию, анализируя свое положение в конкурсном списке (отсортированном по сумме баллов ЕГЭ) и по количеству доступных бюджетных мест на данный УГСН. Если он не попадает в доступные места, то принимется решение о полной смене текущего ВУЗа со всеми заявлениями в нем или о поиске другого ВУЗа, в который можно положить заявление на такой же УГСН.

Оценивание ситуаций производится до тех пор, пока можно подавать новые заявления. Затем абитуриент должен положить оригинал аттестата к одному из своих заявлений. Искомое заявление выбирается исходя из престижности ВУЗа – абитуриенты стремятся попасть в наиболее престижные места. Если выбор проиводится внутри одного ВУЗа на разных УГСН, то сравниваются величины престижности данных УГСН. 

На этапе выбора заявления для подачи оригинала аттестата возможна ситуация, когда абитуриент не имеет ни одного заявления с копией аттестата, соответственно он не может претендовать на возможность обучения в ВУЗе.

\subsection{Разработка метод прогнозирования итогов приема в ВУЗы}

/// IDEF0 ///

Первым этапом моделирования явлется базовое распределение абитуриентов по УГСН в ВУЗах на основе среднего балла по УГСН в прошлом году. Абитуриенты просматривают ВУЗы, начиная с самого престижного, проверяют наличие интересующего УГСН, какие результаты ЕГЭ необходимы для подачи заявления, сравнивают со своими результатами (средним баллом по предметам, необходимым для поступления на данный УГСН) и кладут копию аттестата к заявлению.

Если абитуриент не попадает в доступные бюджетные места на выбранном УГСН, то 

Если студент уже подал заявления в максимальное число вузов, то если заявлений в данном ВУЗе на УГСН больше 1
2.1.1) проверяем по простому соотношению: если кол-во УГСН, на которые студент проходит в данном ВУЗе больше или равно кол-ву на которые не проходит в данном ВУЗе, то заявление не трогаем, иначе ищем другой ВУЗ для всех УГСН, на которые была подана заявка в текущем ВУЗе(из которого убраем все эти заявления)
(поиск нового ВУЗа идет по запомненному ид УГСН, на который надо подать заявление в другом вузе, шансы оцениваются по позиции в конкурсном списке для данного УГСН)

2.1.2) Если в данном вузе есть заявление только на 1 УГСН, то студент забирает копию своего заявления и кладет в другой универ(просмотр опять же в порядке престижности(среднего балла универа)), где есть места на интересующем его УГСН.  
(поиск нового ВУЗа идет по запомненному ид УГСН, на который надо подать заявление в другом вузе, шансы оцениваются по позиции в конкурсном списке для данного УГСН)

2.2) Если студент еще может подать заявления в другой ВУЗ(кол-во использованных ВУЗов меньше 5), то начинается поиск (поиск нового ВУЗа идет по запомненному ид УГСН, на который надо подать заявление в другом вузе, шансы оцениваются по позиции в конкурсном списке для данного УГСН)

При поиске нового места для заявления есть риск, что абитуриент никуда не сможет положить заявление по данному УГСН (не найдет подходящее по условию)

3) Подача оригиналов. Студенты, не имеющие копии заявлений ни в каком вузе пропускаются. Просматриваем каждый УГСН ВУЗа, на который подал заявление студент и смотрим проходит ли он в текущий момент по бюджетным местам среди других студентов с оригиналом. Если проходит, то
3.1) Смотрим есть ли ранее запомненная заявка на УГСН, на который студент тоже проходит по оригиналу, сравниваем значения престижности ВУЗа, если ВУЗы одинаковые, сравниваем значения престижности УГСН (выбираем на какой УГСН круче всего попасть)

3.1.1) Если ранее сохраненной заявки нет, то сохраняем данную заявку как наилучшую на текущей итерации 

4) Все студенты положили оригиналы, теперь смотрим конкурсный список каждого УГСН, фильтруем по оригиналам заявлений и берем n студентов на доступные бюджетные места и вычисляем минимальный, максимальный, средний балл по данному УГСН и ВУЗу в целом.


























\pagebreak