\section*{РЕФЕРАТ}

Расчетно-пояснительная записка \pageref{LastPage} с., \totalfigures\ рис., \totaltables\ табл., 19 ист., 1 прил.

Ключевые слова: приемная кампания, абитуриенты, механизмы зачисления, ЕГЭ, агентное моделирование.


Объектом разработки является метод прогнозирования итогов приёма в ВУЗы России.

Цель работы – разработать и реализовать метод прогнозирования итогов приёма в ВУЗы России на основе агентного моделирования.

Для достижения поставленной цели необходимо решить следующие задачи:

\begin{itemize}[leftmargin=1.6\parindent]
	\item[---] проанализировать правила организации приёма в ВУЗы России;
	\item[---] получить информацию и статистику о ВУЗах России;
	\item[---] получить информацию по УГСН и связанных с ними предметам ЕГЭ;
	\item[---] проанализировать полученную информацию, выделить наиболее важные факторы для разработки модели поведения абитуриентов;
	\item[---] проанализировать существующие методы и механизмы зачисления абитуриентов;
	\item[---] разработать имитационную модель поведения абитуриентов;
	\item[---] разработать метод прогнозирования итогов приёма в ВУЗы, используя модель поведения и статистические данные;
	\item[---] программно реализовать разработанный метод.
\end{itemize}

\pagebreak