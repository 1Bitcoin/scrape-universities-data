\section{Технологическая часть}

\subsection{Реализация алгоритма поиска общего названия ВУЗа}

Для сбора и преобразования статистики о российских ВУЗах в машино-читаемый формат и записи в базу данных, с целью дальнейшей обработки и анализа, был использован Kotlin — статически типизированный высокоуровневый язык программирования\cite{Kotlin}, а для работы с базой данных свободная объектно-реляционная система управления базами данных PostgreSQL\cite{postgresql}.

В процессе анализа статистических данных было обнаружено, что именования ВУЗов на аналитических ресурсах различны, что затрудняет их объединение, поскольку название ВУЗа в данном случае является его уникальным идентификатором. Для унификации названий ВУЗов была использована поисковая система от компании Яндекс. С помощью протокола http был отправлен get-запрос к поисковой системе, где в качестве параметра передавалось название вуза с информационного источника. Анализируя полученный в виде html страницы ответ, можно получить название ВУЗа в интерпретации поисковой системы. В таблице \ref{list:namevuz} представлен пример такой интерпретации.

\begin{table}[!hbp]
	\caption{\label{list:namevuz}Интерпретация названия ВУЗа поисковой системой Яндекс.}
	\begin{center}
\begin{tabular}{|c|c|}
\hline
\begin{tabular}[c]{@{}c@{}}Название \\ ВУЗа \\ на первом\\ ресурсе\end{tabular} & Моск. гос. техн. ун-т. им. Н.Э. Баумана                                                                                                                                                                                                                      \\ \hline
\begin{tabular}[c]{@{}c@{}}Название \\ ВУЗа\\ на втором\\ ресурсе\end{tabular}  & \begin{tabular}[c]{@{}c@{}}федеральное государственное бюджетное \\ образовательное учреждение высшего образования \\ «Московский государственный технический \\ университет имени Н.Э.Баумана (национальный \\ исследовательский университет)»\end{tabular} \\ \hline
\begin{tabular}[c]{@{}c@{}}Общее \\ название\end{tabular}                       & \begin{tabular}[c]{@{}c@{}}Московский государственный технический \\ университет им. Н.Э. Баумана\end{tabular}                                                                                                                                               \\ \hline
\end{tabular}
\end{center}
\end{table}

Однако при обработке названий филиалов, практически во всех случаях результатом было название головного ВУЗа, что являлось ошибочным.  

\pagebreak